\documentclass[]{article}

\usepackage{amsmath, amssymb, amsthm}
\usepackage{geometry}
\usepackage{hyperref, graphicx}

%library theorem environment
\theoremstyle{plain}% default
\newtheorem{thm}{Theorem}[section]
\newtheorem{lem}[thm]{Lemma}    
\newtheorem{prop}[thm]{Proposition}
\newtheorem*{cor}{Corollary}
\newtheorem*{KL}{Klein’s Lemma}
\theoremstyle{definition}
\newtheorem{defn}{Definition}[section]
\newtheorem{conj}{Conjecture}[section]
\newtheorem{exmp}{Example}[section]
\theoremstyle{remark}
\newtheorem*{rem}{Remark}
\newtheorem*{note}{Note}
\newtheorem{case}{Case}
%opening
\title{Optimal Multiplicative partitions: Number vs Individual size}
\author{Daniel Eduardo Ruiz C. \\ \texttt{druizc005@alumno.uaemex.mx}}
\date{\today}

\begin{document}

\maketitle


\begin{abstract}
	This article introduces a function $P_{\pi}(n)$ defined on positive integers, representing the maximum number $k$ of factors in a multiplicative partition of $n$ where each factor $d_i$ must satisfy $d_i \ge k$. We analyze the local behavior of $P_{\pi}(n)$ by defining three sequences based on whether $P_{\pi}(m) > P_{\pi}(m+1)$ (sequence of decrease, $d_j$), $P_{\pi}(m) = P_{\pi}(m+1)$ (sequence of equality, $e_j$), or $P_{\pi}(m) < P_{\pi}(m+1)$ (sequence of increase, $i_j$). We conjecture that these sequences possess asymptotic densities, denoted $C^{(-)}$, $C^{(0)}$, and $C^{(+)}$, respectively. This relies on the observation that the largest sequence terms $s_M$ up to a large integer $M$ are very close to $M$. Empirical evidence for $M$ up to $9 \times 10^5$ suggests that these densities converge, with $C^{(-)} \approx C^{(+)} \approx 0.373$ and $C^{(0)} \approx 0.252$. Furthermore these three local behaviors comprehensively describe the transitions of $P_{\pi}(n)$ across the integers. The study provides statistical support for these conjectures and discusses properties of the sequences and the function $P_{\pi}(n)$.
	
\end{abstract}

\tableofcontents

\section{Principal Definitions}

Multiplicative partitions, which are ways of expressing an integer as a product of integer factors, constitute a classical area of study in number theory. This paper focuses on a particular type of constrained multiplicative partition, characterized by a function denoted $P_{\pi}(n)$. The function $P_{\pi}(n)$ is defined as the largest possible number of factors in a multiplicative partition of $n$ under the specific constraint that every factor must be at least as large as the total count of factors in that partition. As noted, "this function is related to the multiplicative structure of n, and the growth of $P_{\pi}(n)$ is irregular and influenced by the density of its divisors."

\subsection{Partition Function}
\begin{defn}[The function $P_{\pi}(n)$]
	\label{defn:part_func}
	Let $n$ be a positive integer. The function $P_{\pi}(n)$ is defined as the largest positive integer $k$ such that $n$ can be written as a product of $k$ integer factors $d_1, d_2, \ldots, d_k$, i.e., $n = d_1 \cdot d_2 \cdot \ldots \cdot d_k$, where each factor $d_i$ satisfies the condition $d_i \ge k$.
	
	Equivalently, $P_{\pi}(n)$ is the maximum size $k$ of a multi-set of integers $A = \{d_1, d_2, \ldots, d_k\}$ such that:
	\begin{enumerate}
		\item The product of the elements of $A$ is $n$: $\prod_{i=1}^k d_i = n$.
		\item Each element $d_i \in A$ is greater than or equal to the size of the multi-set $A$: $d_i \ge k$ for all $i=1, \ldots, k$.
	\end{enumerate}
	For $n=1$, the only multi-set is $A=\{1\}$. The size of this multi-set is $k=1$. The single factor $d_1=1$ satisfies the condition $d_1 \ge k$ (since $1 \ge 1$). Thus, $f(1)=1$.
\end{defn}
\begin{figure}[h]
	\caption{Partition Function for $n$ up to 200}
	\centering
	\includegraphics[width=0.9\textwidth]{graph/partition_function_200.png}
\end{figure}
\begin{exmp}
	The following examples illustrate the calculation of $P_{\pi}(n)$:
	\begin{itemize}
		\item For $n=10$:
		If $k=1$, the partition is $\{10\}$. Since $10 \ge 1$, this is valid.
		If $k=2$, we need factors $d_1, d_2$ such that $d_1 d_2 = 10$ and $d_1, d_2 \ge 2$. The partition $\{2,5\}$ satisfies these conditions ($2 \ge 2$ and $5 \ge 2$). So $k=2$ is possible.
		If $k=3$, we need factors $d_1, d_2, d_3$ such that $d_1 d_2 d_3 = 10$ and $d_1, d_2, d_3 \ge 3$. The smallest possible product of three integers, each at least 3, is $3 \cdot 3 \cdot 3 = 27$. Since $27 > 10$, $k=3$ is not possible for $n=10$.
		Thus, the maximum $k$ is 2, so $f(10)=2$.
		
		\item For $n=11$:
		If $k=1$, the partition is $\{11\}$. Since $11 \ge 1$, this is valid.
		If $k=2$, we need factors $d_1, d_2$ such that $d_1 d_2 = 11$ and $d_1, d_2 \ge 2$. As 11 is prime, its only positive integer factors are 1 and 11. The partition $\{1,11\}$ has $d_1=1$, which does not satisfy $d_1 \ge 2$. Thus, $k=2$ is not possible.
		Therefore, $f(11)=1$.
		
		\item For $n=63$:
		The problem states that $f(63)=3$. This is supported by the partition $\{3,3,7\}$. Here, $k=3$, and all factors ($3,3,7$) are greater than or equal to $k=3$.
		To confirm $k$ cannot be 4, we would require four factors $d_1, d_2, d_3, d_4$ each at least 4. Their product would be at least $4^4 = 256$, which is greater than 63. So $f(63)=3$.
		
		\item For $n=64$:
		The problem states that $f(64)=3$. This is supported by the partition $\{4,4,4\}$. Here, $k=3$, and all factors ($4,4,4$) are greater than or equal to $k=3$.
		Similarly, $k=4$ would require factors $\ge 4$, leading to a product of at least $4^4=256 > 64$. So $f(64)=3$.
	\end{itemize}
\end{exmp}
\subsection{Local Behavior Sequences}
Using the function $P_{\pi}(n)$, we define three sequences based on the relationship between $P_{\pi}(m)$ and $P_{\pi}(m+1)$.

\begin{defn}[Sequence of Decrease $(d_j)$]
	The sequence $(d_j)_{j \ge 1}$ consists of all positive integers $m$, listed in increasing order, such that $P_{\pi}(m) > P_{\pi}(m+1)$.
\end{defn}

\begin{figure}[h]
	\caption{Sequence of Decrease for $d_j$ up to $10^5$}
	\centering
	\includegraphics[width=0.9\textwidth]{graph/decrease_sequence_10_5.png}
\end{figure}

\begin{defn}[Sequence of Equality $(e_j)$]
	The sequence $(e_j)_{j \ge 1}$ consists of all positive integers $m$, listed in increasing order, such that $P_{\pi}(m) = P_{\pi}(m+1)$.
\end{defn}
\begin{figure}[h]
	\caption{Sequence of Equality for $e_j$ up to $10^5$}
	\centering
	\includegraphics[width=0.9\textwidth]{graph/equality_sequence_10_5.png}
\end{figure}
\begin{defn}[Sequence of Increase $(i_j)$]
	The sequence $(i_j)_{j \ge 1}$ consists of all positive integers $m$, listed in increasing order, such that $P_{\pi}(m) < P_{\pi}(m+1)$.
\end{defn}
\begin{figure}[h]
	\caption{Sequence of Increase for $i_j$ up to $10^5$}
	\centering
	\includegraphics[width=0.9\textwidth]{graph/increase_sequence_10_5.png}
\end{figure}
\begin{exmp}
	The behavior of $P_{\pi}(n)$ at consecutive integers determines membership in these sequences:
	\begin{itemize}
		\item For $m=10$: We have $P_{\pi}(10)=2$ and $P_{\pi}(11)=1$. Since $P_{\pi}(10) > P_{\pi}(11)$ (i.e., $2 > 1$), the integer $10$ is a term in the sequence $(d_j)$.
		
		\item For $m=63$: We have $P_{\pi}(63)=3$ and $P_{\pi}(64)=3$. Since $P_{\pi}(63) = P_{\pi}(64)$ (i.e., $3 = 3$), the integer $63$ is a term in the sequence $(e_j)$.
		
		\item To illustrate the sequence $(i_j)$, consider $m=7$.
		$P_{\pi}(7)=1$ (since 7 is prime, only partition is $\{7\}$, $k=1$, $7 \ge 1$).
		For $P_{\pi}(8)$: If $k=1$, $\{8\}$ is valid. If $k=2$, factors $d_1, d_2 \ge 2$. $\{2,4\}$ works ($2\ge 2, 4\ge 2$). If $k=3$, factors $d_1,d_2,d_3 \ge 3$. Smallest product $3^3=27 > 8$. So $P_{\pi}(8)=2$.
		Since $P_{\pi}(7) < P_{\pi}(8)$ (i.e., $1 < 2$), the integer $7$ is a term in the sequence $(i_j)$.
	\end{itemize}
\end{exmp}
\subsection{Initial Terms of $P_{\pi}(n)$ and Derived Sequences}
\label{subsec:initial_terms}
This section is reserved for listing the initial terms of the partition function $P_{\pi}(n)$ and the derived sequences $(d_j)$, $(e_j)$, and $(i_j)$.

\vspace{5mm}
\noindent \textbf{Initial values of $P_{\pi}(n)$:}
% TODO: Populate with P_pi(n) for n = 1, 2, ..., N (e.g., N=20 or 30)
% Example: P_pi(1)=1, P_pi(2)=1, P_pi(3)=1, P_pi(4)=1, P_pi(5)=1, P_pi(6)=1, P_pi(7)=1, P_pi(8)=2, ...


\noindent \textbf{Initial terms of Sequence of Decrease $(d_j)$:}
4, 6, 10, 12, 16, 18, 22, 27, 28, 30, 36, 40, 42, 45, 46, 48, 52, 54, 58, 60, 64, 66, 70, 72, 75, 78, 81, 82, 84, 88, 90, 96, 100, 102, 105, 106, 108, 112, 117, 120 \dots

\noindent \textbf{Initial terms of Sequence of Equality $(e_j)$:}
1, 2, 8, 9, 14, 15, 20, 21, 24, 25, 32, 33, 34, 38, 39, 49, 50, 51, 55, 56, 57, 63, 65, 68, 69, 76, 77, 80, 85, 86, 87, 91, 92, 93, 94, 99, 110, 114, 115, 118, 121 \dots

\noindent \textbf{Initial terms of Sequence of Increase $(i_j)$:}
3, 5, 7, 11, 13, 17, 19, 23, 26, 29, 31, 35, 37, 41, 43, 44, 47, 53, 59, 61, 62, 67, 71, 73, 74, 79, 83, 89, 95, 97, 98, 101, 103, 104, 107, 109, 111, 113, 116, 119 \dots


\section {Convergence Conjecture}
\begin{conj}[Asymptotic Densities of Local Behaviors]
	\label{conj:ratios}
	Let $M$ be a fixed integer and let:
	$$ j^{(-)}_M = \max \{ j :  d_j\le M\},$$
	$$ j^{(0)}_M = \max \{ j :  e_j\le M\},$$
	and
	$$ j^{(+)}_M = \max \{ j :  i_j\le M\}.$$
	The elements of the three sequences are conjectured to satisfy
	$$   \forall s_M \in \{d_{j^{(-)}_M},e_{j^{(0)}_M},i_{j^{(+)}_M}\}, \quad |M-s_M| \ll M. $$
	It is further conjectured the asymptotic densities of these sequences exist and are given by:
	$$ C^{(\pm)} = \lim_{M \to \infty} \frac{j^{(-)}_M}{d_{j^{(-)}_M}} = \lim_{M \to \infty}\frac{j^{(+)}_M}{i_{j^{(+)}_M}}, \quad C^{(0)} = \lim_{M \to \infty} \frac{j^{(0)}_M}{e_{j^{(0)}_M}},$$
	and consequently
	$$ 1 = 2C^{(\pm)} +C^{(0)}.$$
\end{conj}
\begin{rem}
	The condition $ |M-s_M| \ll M $ for $ s_M \in \{d_{j^{(-)}_M},e_{j^{(0)}_M},i_{j^{(+)}_M}\} $ signifies that for a sufficiently large integer $M$, the largest terms $d_{j^{(-)}_M}$, $e_{j^{(0)}_M}$, and $i_{j^{(+)}_M}$ of the respective sequences that are less than or equal to $M$ are indeed very close to $M$. This proximity ensures that the counts $j^{(-)}_M, j^{(0)}_M, j^{(+)}_M$ (i.e., the number of terms up to $M$) are representative of the density of these sequence terms across the interval $[1, M]$. Consequently, the ratios $C^{(-)} = j^{(-)}_M/d_{j^{(-)}_M}$, $C^{(0)} = j^{(0)}_M/e_{j^{(0)}_M}$, and $C^{(+)} = j^{(+)}_M/i_{j^{(+)}_M}$ (often approximated as $j^{(-)}_M/M$, $j^{(0)}_M/M$, $j^{(+)}_M/M$ due to $s_M \approx M$) become meaningful estimates of the true asymptotic densities. Table \ref{tab:statistical_differences_analysis} provides empirical support for this condition, demonstrating that the relative differences $|M-s_M|/M$ are consistently small for large $M$.
\end{rem}
\begin{figure}[h]
	\caption{Bar Chart of Distribution by Levels for values up to $10^5$}
	\centering
	\includegraphics[width=0.9\textwidth]{graph/level_frequencies_10_5.png}
\end{figure}
\subsection{Empirical Observations and Sequence Densities}
The conjectured limits $C^{(-)}, C^{(0)},$ and $C^{(+)}$ can be interpreted as the asymptotic densities of integers $m$ for which $P_{\pi}(m) > P_{\pi}(m+1)$, $P_{\pi}(m) = P_{\pi}(m+1)$, and $P_{\pi}(m) < P_{\pi}(m+1)$, respectively.
Analysis of the data in Table \ref{tab:statistical_conjecture_analysis} for values of $M$ up to $9 \times 10^5$ reveals the following trends for the estimated densities:
\begin{itemize}
	\item The density $C^{(-)}$ for the sequence of decrease $(d_j)$ stabilizes around $0.3736$.
	\item The density $C^{(0)}$ for the sequence of equality $(e_j)$ stabilizes around $0.2525$.
	\item The density $C^{(+)}$ for the sequence of increase $(i_j)$ stabilizes around $0.3737$.
\end{itemize}
These numerical results suggest that integers $m$ marking a decrease ($P_{\pi}(m) > P_{\pi}(m+1)$) and those marking an increase ($P_{\pi}(m) < P_{\pi}(m+1)$) in the $P_{\pi}$ function occur with nearly identical and the highest frequencies. Integers $m$ where $P_{\pi}(m) = P_{\pi}(m+1)$ are less frequent. The sum $2C^{(\pm)} + C^{(0)}$ (approximating $2 \times 0.3736 + 0.2525 \approx 0.7472 + 0.2525 = 0.9997$ for the largest $M$ values) is consistently close to $1$, as detailed in Table \ref{tab:statistical_conjecture_relation}. This empirically validates the conjecture that these three types of local behavior partition the set of positive integers, and leads to the two following conclusions:
\begin{itemize}
	\item The condition $|M-s_M| \ll M$ (Table \ref{tab:statistical_differences_analysis}) holds robustly, indicating that the chosen $M$ values do not terminate prematurely within long stretches devoid of sequence terms. This supports the reliability of density calculations.
	\item The convergence of $C^{(-)}, C^{(0)}, C^{(+)}$ appears relatively smooth with increasing $M$.
\end{itemize}
Further investigation could explore the distribution of lengths of consecutive runs of $d_j, e_j,$ or $i_j$ type integers, or analyze the behavior of these densities over different scales and special values of $M$.

\begin{table}[h]
	\centering
	\caption{Statistical Convergence Analysis of the Conjecture \ref{conj:ratios} for Various Values of $M$ up to $10^6$}
	\label{tab:statistical_conjecture_analysis}
	\resizebox{\textwidth}{!}{%
		\begin{tabular}{|c|c|c|c|c|c|c|c|c|c|}
			\hline
			$M$ & $j^{(-)}_M$ & $d_{j^{(-)}_M}$ & $j^{(0)}_M$ & $e_{j^{(0)}_M}$ & $j^{(+)}_M$ & $i_{j^{(+)}_M}$ & $C^{(-)}$ & $C^{(0)}$ & $C^{(+)}$ \\
			\hline
			38838 & 14226 & 38838 & 10379 & 38837 & 14233 & 38835 & 0.366291 & 0.267245 & 0.366499 \\
			\hline
			49904 & 18295 & 49900 & 13326 & 49904 & 18283 & 49903 & 0.366633 & 0.267033 & 0.366371 \\
			\hline
			63946 & 23496 & 63945 & 16964 & 63946 & 23486 & 63944 & 0.367441 & 0.265286 & 0.367290 \\
			\hline
			64480 & 23685 & 64480 & 17119 & 64478 & 23676 & 64479 & 0.367323 & 0.265501 & 0.367189 \\
			\hline
			75000 & 27588 & 75000 & 19830 & 74997 & 27582 & 74999 & 0.367840 & 0.264411 & 0.367765 \\
			\hline
			1e+05 & 36867 & 100000 & 26270 & 99996 & 36863 & 99999 & 0.368670 & 0.262711 & 0.368634 \\
			\hline
			1e+05 & 45690 & 123805 & 32436 & 123806 & 45681 & 123807 & 0.369048 & 0.261991 & 0.368969 \\
			\hline
			1e+05 & 46103 & 124944 & 32731 & 124941 & 46110 & 124943 & 0.368989 & 0.261972 & 0.369048 \\
			\hline
			2e+05 & 55394 & 150000 & 39201 & 149998 & 55405 & 149999 & 0.369293 & 0.261343 & 0.369369 \\
			\hline
			2e+05 & 80409 & 217065 & 56321 & 217066 & 80337 & 217067 & 0.370437 & 0.259465 & 0.370102 \\
			\hline
			2e+05 & 92675 & 250000 & 64738 & 249998 & 92587 & 249999 & 0.370700 & 0.258954 & 0.370349 \\
			\hline
			4e+05 & 130047 & 350000 & 89896 & 349995 & 130057 & 349999 & 0.371563 & 0.256849 & 0.371592 \\
			\hline
			4e+05 & 149291 & 401536 & 102901 & 401532 & 149345 & 401537 & 0.371800 & 0.256271 & 0.371933 \\
			\hline
			5e+05 & 199947 & 536700 & 136788 & 536694 & 199965 & 536699 & 0.372549 & 0.254871 & 0.372583 \\
			\hline
			6e+05 & 239123 & 641396 & 163090 & 641398 & 239185 & 641387 & 0.372816 & 0.254273 & 0.372918 \\
			\hline
			7e+05 & 262535 & 703872 & 178696 & 703870 & 262642 & 703873 & 0.372987 & 0.253876 & 0.373138 \\
			\hline
			8e+05 & 279870 & 750000 & 190149 & 749997 & 279981 & 749999 & 0.373160 & 0.253533 & 0.373308 \\
			\hline
			9e+05 & 338604 & 906176 & 228881 & 906175 & 338692 & 906177 & 0.373663 & 0.252579 & 0.373759 \\
			\hline
			
		\end{tabular}%
	}
\end{table}

\begin{table}[h]
	\centering
	\caption{Statistical Conjecture \ref{conj:ratios} Relation Analysis: $2C^{(\pm)} + C^{(0)} = 1$}
	\label{tab:statistical_conjecture_relation}
	\resizebox{\textwidth}{!}{%
		\begin{tabular}{|c|c|c|c|c|c|c|c|}
			\hline
			$M$ & $C^{(-)}$ & $C^{(0)}$ & $C^{(+)}$ & $2C^{(-)}$ & $2C^{(-)} + C^{(0)}$ & $2C^{(+)} + C^{(0)}$ & $\left|1 - (2C^{(-)} + C^{(0)})\right|$ \\
			\hline
			38838 & 0.366291 & 0.267245 & 0.366499 & 0.732581 & 0.999827 & 1.000244 & 0.000173 \\
			\hline
			49904 & 0.366633 & 0.267033 & 0.366371 & 0.733267 & 1.000299 & 0.999774 & 0.000299 \\
			\hline
			63946 & 0.367441 & 0.265286 & 0.367290 & 0.734882 & 1.000168 & 0.999867 & 0.000168 \\
			\hline
			64480 & 0.367323 & 0.265501 & 0.367189 & 0.734646 & 1.000148 & 0.999880 & 0.000148 \\
			\hline
			75000 & 0.367840 & 0.264411 & 0.367765 & 0.735680 & 1.000091 & 0.999940 & 0.000091 \\
			\hline
			1e+05 & 0.368670 & 0.262711 & 0.368634 & 0.737340 & 1.000051 & 0.999978 & 0.000051 \\
			\hline
			1e+05 & 0.369048 & 0.261991 & 0.368969 & 0.738096 & 1.000087 & 0.999929 & 0.000087 \\
			\hline
			1e+05 & 0.368989 & 0.261972 & 0.369048 & 0.737979 & 0.999950 & 1.000068 & 0.000050 \\
			\hline
			2e+05 & 0.369293 & 0.261343 & 0.369369 & 0.738587 & 0.999930 & 1.000082 & 0.000070 \\
			\hline
			2e+05 & 0.370437 & 0.259465 & 0.370102 & 0.740875 & 1.000340 & 0.999670 & 0.000340 \\
			\hline
			2e+05 & 0.370700 & 0.258954 & 0.370349 & 0.741400 & 1.000354 & 0.999653 & 0.000354 \\
			\hline
			4e+05 & 0.371563 & 0.256849 & 0.371592 & 0.743126 & 0.999975 & 1.000034 & 0.000025 \\
			\hline
			4e+05 & 0.371800 & 0.256271 & 0.371933 & 0.743600 & 0.999871 & 1.000138 & 0.000129 \\
			\hline
			5e+05 & 0.372549 & 0.254871 & 0.372583 & 0.745098 & 0.999969 & 1.000038 & 0.000031 \\
			\hline
			6e+05 & 0.372816 & 0.254273 & 0.372918 & 0.745633 & 0.999906 & 1.000109 & 0.000094 \\
			\hline
			7e+05 & 0.372987 & 0.253876 & 0.373138 & 0.745974 & 0.999850 & 1.000153 & 0.000150 \\
			\hline
			8e+05 & 0.373160 & 0.253533 & 0.373308 & 0.746320 & 0.999853 & 1.000150 & 0.000147 \\
			\hline
			9e+05 & 0.373663 & 0.252579 & 0.373759 & 0.747325 & 0.999904 & 1.000098 & 0.000096 \\
			\hline
			
		\end{tabular}%
	}
\end{table}

\begin{table}[h]
	\centering
	\caption{Statistical Analysis of Differences $|M - s_M|$ for Convergence in Conjecture \ref{conj:ratios}}
	\label{tab:statistical_differences_analysis}
	\resizebox{\textwidth}{!}{%
		\begin{tabular}{|c|c|c|c|c|c|c|}
			\hline
			$M$ & $|M - d_{j^{(-)}_M}|$ & $|M - e_{j^{(0)}_M}|$ & $|M - i_{j^{(+)}_M}|$ & $\frac{|M - d_{j^{(-)}_M}|}{M}$ & $\frac{|M - e_{j^{(0)}_M}|}{M}$ & $\frac{|M - i_{j^{(+)}_M}|}{M}$ \\
			\hline
			38838 & 0 & 1 & 3 & 0.000000 & 0.000026 & 0.000077 \\
			\hline
			49904 & 4 & 0 & 1 & 0.000080 & 0.000000 & 0.000020 \\
			\hline
			63946 & 1 & 0 & 2 & 0.000016 & 0.000000 & 0.000031 \\
			\hline
			64480 & 0 & 2 & 1 & 0.000000 & 0.000031 & 0.000016 \\
			\hline
			75000 & 0 & 3 & 1 & 0.000000 & 0.000040 & 0.000013 \\
			\hline
			1e+05 & 0 & 4 & 1 & 0.000000 & 0.000040 & 0.000010 \\
			\hline
			1e+05 & 2 & 1 & 0 & 0.000016 & 0.000008 & 0.000000 \\
			\hline
			1e+05 & 0 & 3 & 1 & 0.000000 & 0.000024 & 0.000008 \\
			\hline
			2e+05 & 0 & 2 & 1 & 0.000000 & 0.000013 & 0.000007 \\
			\hline
			2e+05 & 2 & 1 & 0 & 0.000009 & 0.000005 & 0.000000 \\
			\hline
			2e+05 & 0 & 2 & 1 & 0.000000 & 0.000008 & 0.000004 \\
			\hline
			4e+05 & 0 & 5 & 1 & 0.000000 & 0.000014 & 0.000003 \\
			\hline
			4e+05 & 1 & 5 & 0 & 0.000002 & 0.000012 & 0.000000 \\
			\hline
			5e+05 & 0 & 6 & 1 & 0.000000 & 0.000011 & 0.000002 \\
			\hline
			6e+05 & 2 & 0 & 11 & 0.000003 & 0.000000 & 0.000017 \\
			\hline
			7e+05 & 1 & 3 & 0 & 0.000001 & 0.000004 & 0.000000 \\
			\hline
			8e+05 & 0 & 3 & 1 & 0.000000 & 0.000004 & 0.000001 \\
			\hline
			9e+05 & 1 & 2 & 0 & 0.000001 & 0.000002 & 0.000000 \\
			\hline
			
		\end{tabular}%
	}
\end{table}
\section{Conclusion and Future Work}
\label{sec:conclusion}

The function $P_{\pi}(n)$ and its associated local behavior sequences $(d_j)$, $(e_j)$, and $(i_j)$ provide a unique framework for analyzing multiplicative partitions constrained by the number of their parts. The empirical data presented strongly supports the Convergence Conjecture, suggesting that these three types of local transitions partition the set of positive integers with stable asymptotic densities, where $C^{(-)} \approx C^{(+)} \approx 0.373$ and $C^{(0)} \approx 0.252$ based on current estimates.

This study opens up several promising directions for future research:

\subsection{Properties of $P_{\pi}(n)$ and Derived Sequences}
\begin{itemize}
	\item \textbf{Analytical Understanding of $P_{\pi}(n)$:} Developing analytical bounds (beyond trivial ones), determining the average or maximal order of $P_{\pi}(n)$, or finding an asymptotic formula for $P_{\pi}(n)$ would be significant theoretical advancements.
	\item \textbf{Characterizing Sequence Members:} A deeper dive into the number-theoretic properties (e.g., prime factorization structure, divisibility, values of other arithmetic functions like $\Omega(n)$ or $\sigma_0(n)$) of integers within each sequence $(d_j)$, $(e_j)$, $(i_j)$ could illuminate underlying reasons for their classification.
	\item \textbf{Investigating Runs and Patterns:} Studying the distribution of lengths of consecutive integers belonging to the same sequence type (e.g., a long run of $m$ where $P_{\pi}(m) = P_{\pi}(m+1)$) could reveal insights into the local stability and correlational structure of $P_{\pi}(n)$.
\end{itemize}

\subsection{Level Distributions and Conditional Frequencies}
\begin{itemize}
	\item \textbf{Behavior within $P_{\pi}(n)=k$ Strata:} For a fixed integer $k \ge 1$, let $S_k = \{ m : P_{\pi}(m) = k \}$. Analyzing the conditional frequencies of $m \in S_k$ also belonging to $(d_j)$, $(e_j)$, or $(i_j)$ could be very revealing. For example, how does the proportion of $m \in S_k$ that are points of increase (i.e., $m \in (i_j)$) change as $k$ varies? This could shed light on how the function typically transitions away from a certain level $k$.
	\item \textbf{Specific Case $P_{\pi}(n)=1$:} Integers $m$ with $P_{\pi}(m)=1$ (which include all primes, and other numbers like $1, 4, 6, 9, \dots$) cannot be in $(d_j)$ because $P_{\pi}(m+1)$ must be at least $1$. Thus, these $m$ are either in $(e_j)$ (if $P_{\pi}(m+1)=1$) or $(i_j)$ (if $P_{\pi}(m+1)>1$). The relative frequencies of these outcomes for the subset of integers with $P_{\pi}(m)=1$ would be a focused area of study.
\end{itemize}

\subsection{Analysis of Convergence and the $|M-s_M|$ Gap}
\begin{itemize}
	\item \textbf{The Nature of the Gap $|M-s_M|$:} Table \ref{tab:statistical_differences_analysis} indicates that for $s_M \in \{d_{j^{(-)}_M}, e_{j^{(0)}_M}, i_{j^{(+)}_M}\}$, the absolute difference $|M-s_M|$ is small. The maximum such difference observed in the provided table is 11. A key question is whether this difference $|M-s_M|$ remains bounded as $M \to \infty$, or if it grows, how slowly (e.g., $\log M$, $\log \log M$). This is crucial for the rigor of using $M$ as the denominator in density calculations.
	\item \textbf{Rate of Convergence of Densities:} Quantifying the speed at which the ratios $j^{(-)}_M/M$, $j^{(0)}_M/M$, and $j^{(+)}_M/M$ approach their respective limits $C^{(-)}$, $C^{(0)}$, and $C^{(+)}$ could provide further insights, possibly involving error terms related to $M$.
\end{itemize}

\subsection{Further Explorations and Generalizations}
\begin{itemize}
	\item \textbf{Modifying Partition Constraints:} One could explore variations of the $P_{\pi}(n)$ function by altering the fundamental constraint $d_i \ge k$. For instance, what if $d_i \ge c \cdot k$ for some constant $c \neq 1$, or $d_i \ge k-c$, or $d_i \ge f(k)$ for some other slowly growing function $f$? How would such changes impact the existence and values of the corresponding densities?
	\item \textbf{Probabilistic Modeling:} Attempting to model the sequence of differences $P_{\pi}(m+1) - P_{\pi}(m)$ using a probabilistic framework (e.g., a type of random walk or a Markov chain on the values of $P_{\pi}(n)$) might offer a path to theoretically derive or explain the observed densities.
	\item \textbf{Relationship to Highly Composite Numbers:} Since $P_{\pi}(n)$ relates to factorizations, exploring its behavior for specific classes of numbers, such as highly composite numbers, primorials, or perfect powers, might reveal interesting patterns or extremal behaviors.
\end{itemize}

Pursuing these research avenues could substantially enhance our comprehension of this particular type of multiplicative partition and the rich statistical patterns that emerge from its local behavior.

\end{document}
